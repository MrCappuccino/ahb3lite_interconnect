\chapter{Interfaces}\label{interfaces}


\section{Global Signals}\label{global-signals}

The common signals are shared between all devices on the AHB bus. The
AHB-Lite Interconnect has Master and Slave AHB-Lite buses and they all
use the global signals.

\begin{longtable}[]{@{}lccl@{}}
\toprule
Port & Size & Direction & Description\tabularnewline
\midrule
\endhead
\texttt{HRESETn} & 1 & Input & Asynchronous active low reset\tabularnewline
\texttt{HCLK} & 1 & Input & System clock input\tabularnewline
\bottomrule
\caption{AMBA3 Global Signals}
\end{longtable}

\subsection{HRESETn}\label{hresetn}

When the active low asynchronous \texttt{HRESETn} input is asserted (`0'), the
core is put into its initial reset state.

\subsection{HCLK}\label{hclk}

\texttt{HCLK} is the system clock. All internal logic operates at the rising edge
of the system clock. All AHB bus timings are related to the rising edge
of \texttt{HCLK}. All Master and Slave ports must operate at the same \texttt{HCLK} clock.

\section{Master Interfaces}\label{master-interfaces}

The Master Ports are regular AHB3-Lite slave interfaces. All signals are
supported. See the AHB-Lite specifications for a complete description of
the signals.

The AHB-Lite Multi-layer Interconnect implements 1 or more interfaces to AHB-Lite masters
as defined by the \texttt{MASTERS} parameter. Therefore the following signals are all
arrays reflecting the number of masters supported.

%\newpage
\begin{longtable}[]{@{}lccl@{}}
\toprule
Port & Size & Direction & Description\tabularnewline
\midrule
\endhead
\texttt{mst\_priority}  & clog\textsubscript{2}(\texttt{MASTERS}) & Input & Master Priority Levels\tabularnewline
\bottomrule
\caption{Master Interface Customisation Port}
\end{longtable}

\textbf{Note:} clog\textsubscript{2}() refers to the System Verilog function by
the same name, defined below, and is used to determine the required bitwidth of a
bus required to represent the defined range of values:

\begin{quote}
\emph{The system function \$clog2 shall return the ceiling of the log
base 2 of the argument (the log rounded up to an integer value). The
argument can be an integer or an arbitrary sized vector value. The
argument shall be treated as an unsigned value, and an argument value of
0 shall produce a result of 0.}
\end{quote}

\begin{longtable}[]{@{}lccl@{}}
\toprule
Port & Size & Direction & Description\tabularnewline
\midrule
\endhead
\texttt{mst\_HSEL}      & 1                     & Input  & Bus Select\tabularnewline
\texttt{mst\_HTRANS}    & 2                     & Input  & Transfer Type\tabularnewline
\texttt{mst\_HADDR}     & \texttt{HADDR\_SIZE}  & Input  & Address Bus\tabularnewline
\texttt{mst\_HWDATA}    & \texttt{HDATA\_SIZE}  & Input  & Write Data Bus\tabularnewline
\texttt{mst\_HRDATA}    & \texttt{HDATA\_SIZE}  & Output & Read Data Bus\tabularnewline
\texttt{mst\_HWRITE}    & 1                     & Input  & Write Select\tabularnewline
\texttt{mst\_HSIZE}     & 3                     & Input  & Transfer Size\tabularnewline
\texttt{mst\_HBURST}    & 3                     & Input  & Transfer Burst Size\tabularnewline
\texttt{mst\_HPROT}     & 4                     & Input  & Transfer Protection Level\tabularnewline
\texttt{mst\_HMASTLOCK} & 1                     & Input  & Transfer Master Lock\tabularnewline
\texttt{mst\_HREADYOUT} & 1                     & Output & Transfer Ready Output\tabularnewline
\texttt{mst\_HREADY}    & 1                     & Input  & Transfer Ready Input\tabularnewline
\texttt{mst\_HRESP}     & 1                     & Input  & Transfer Response\tabularnewline
\bottomrule
\caption{Master Interface AHB-Lite Port}
\end{longtable}

\subsection{mst\_priority}\label{mst_priority}

\texttt{mst\_priority} defines the priority of each master. The width of the bus is calculated as clog\textsubscript{2}(\texttt{MASTERS}). For example, for a system with 4 masters the width of \texttt{mst\_priority} will be 2 bits.

Lowest priority is 0, highest priority is \texttt{MASTERS - 1}.

\subsection{mst\_HSEL}\label{mst_hsel}

The Master Port only responds to other signals on its bus when \texttt{mst\_HSEL} is
asserted (`1'). When \texttt{mst\_HSEL} is negated (`0') the Master Port
considers the bus IDLE and asserts \texttt{HREADYOUT} (`1').

\subsection{mst\_HTRANS}\label{mst_htrans}

\texttt{mst\_HTRANS} indicates the type of the current transfer. It is driven to
the connected slave.

\begin{longtable}[]{@{}clp{9cm}@{}}
\toprule
HTRANS & Type & Description\tabularnewline
\midrule
\endhead
00 & IDLE & No transfer required\tabularnewline
01 & BUSY & Connected master is not ready to accept data, but intents to
continue the current burst.\tabularnewline
10 & NONSEQ & First transfer of a burst or a single
transfer\tabularnewline
11 & SEQ & Remaining transfers of a burst\tabularnewline
\bottomrule
\caption{Transfer Type (HTRANS)}
\end{longtable}


\subsection{mst\_HADDR}\label{mst_haddr}

\texttt{mst\_HADDR} is the address bus. Its size is determined by the \texttt{HADDR\_SIZE}
parameter. It is driven to the connected slave.

\subsection{mst\_HWDATA}\label{mst_hwdata}

\texttt{mst\_HWDATA} is the write data bus. Its size is determined by the
\texttt{HDATA\_SIZE} parameter. It is driven to the connected slave.

\subsection{mst\_HRDATA}\label{mst_hrdata}

\texttt{mst\_HRDATA} is the read data bus. Its size is determined by \texttt{HDATA\_SIZE}
parameter. The connected slave drives it.

\subsection{mst\_HWRITE}\label{mst_hwrite}

\texttt{mst\_HWRITE} is the read/write signal. \texttt{mst\_HWRITE} asserted (`1') indicates a
write transfer. It is driven to the connected slave.

\subsection{mst\_HSIZE}\label{mst_hsize}

\texttt{mst\_HSIZE} indicates the size of the current transfer. It is driven to
the connected slave.

\begin{longtable}[]{@{}ccl@{}}
\toprule
HSIZE & Size & Description\tabularnewline
\midrule
\endhead
\texttt{000} & 8bit    & Byte\tabularnewline
\texttt{001} & 16bit   & Half Word\tabularnewline
\texttt{010} & 32bit   & Word\tabularnewline
\texttt{011} & 64bits  & Double Word\tabularnewline
\texttt{100} & 128bit  & \tabularnewline
\texttt{101} & 256bit  & \tabularnewline
\texttt{110} & 512bit  & \tabularnewline
\texttt{111} & 1024bit & \tabularnewline
\bottomrule
\caption{Transfer Size Values (HSIZE)}
\end{longtable}


\subsection{mst\_HBURST}\label{mst_hburst}

The burst type indicates if the transfer is a single transfer or part of
a burst. It is driven to the connected slave.

\begin{longtable}[]{@{}ccl@{}}
\toprule
HBURST & Type & Description\tabularnewline
\midrule
\endhead
\texttt{000} & SINGLE & Single access\tabularnewline
\texttt{001} & INCR & Continuous incremental burst\tabularnewline
\texttt{010} & WRAP4 & 4-beat wrapping burst\tabularnewline
\texttt{011} & INCR4 & 4-beat incrementing burst\tabularnewline
\texttt{100} & WRAP8 & 8-beat wrapping burst\tabularnewline
\texttt{101} & INCR8 & 8-beat incrementing burst\tabularnewline
\texttt{110} & WRAP16 & 16-beat wrapping burst\tabularnewline
\texttt{111} & INCR16 & 16-beat incrementing burst\tabularnewline
\bottomrule
\caption{Burst Types (HBURST)}
\end{longtable}

\subsection{mst\_HPROT}\label{mst_hprot}

The protection signals provide information about the bus transfer. They
are intended to implement some level of protection. It is driven to the
connected slave.

\begin{longtable}[]{@{}ccl@{}}
\toprule
Bit \# & Value & Description\tabularnewline
\midrule
\endhead
3 & 1 & Cacheable region addressed\tabularnewline
& 0 & Non-cacheable region addressed\tabularnewline
2 & 1 & Bufferable\tabularnewline
& 0 & Non-bufferable\tabularnewline
1 & 1 & Privileged Access\tabularnewline
& 0 & User Access\tabularnewline
0 & 1 & Data Access\tabularnewline
& 0 & Opcode fetch\tabularnewline
\bottomrule
\caption{Protection Signals (HPROT)}
\end{longtable}

\subsection{mst\_HREADYOUT}\label{mst_hreadyout}

When a slave is addressed, the \texttt{mst\_HREADYOUT} indicates that the
addressed slave finished the current transfer. The Interconnect IP
routes the addressed slave's \texttt{HREADY} signal to the master.

When no slave is addressed, the \texttt{mst\_HREADYOUT} signal is generated
locally, inside the Interconnect.

\subsection{mst\_HMASTLOCK}\label{mst_hmastlock}

The master lock signal indicates if the current transfer is part of a
locked sequence, commonly used for Read-Modify-Write cycles. While the
\texttt{mst\_HMASTLOCK} is asserted, the Interconnect IP cannot switch the
addressed slave to another master, even if that master has a higher
priority. Instead the current master retains access to slave until it
releases \texttt{mst\_HMASTLOCK}.

\subsection{mst\_HREADY}\label{mst_hready}

\texttt{mst\_HREADY} indicates the status of the local \texttt{HREADY} on the master's
local bus. It is routed to the \texttt{HREADYOUT} port of the connected slave.

\subsection{mst\_HRESP}\label{mst_hresp}

\texttt{mst\_HRESP} is the transfer response from the connected slave, it can
either be OKAY (`0') or ERROR (`1'). The Interconnect IP routes the
connected slave's \texttt{HRESP} port to \texttt{mst\_HRESP}.

\section{Slave Interface}\label{slave-interface}

The Slave Ports are regular AHB-Lite master interfaces.. All signals are
supported. In addition each Slave Port has a non-standard
\texttt{slv\_HREADYOUT}. See the AHB-Lite specifications for a complete
description of the signals.

The AHB-Lite Multi-layer Interconnect implements 1 or more interfaces to AHB-Lite slaves
as defined by the \texttt{SLAVES} parameter. Therefore the following signals are all
arrays reflecting the number of slaves supported.

\begin{longtable}[]{@{}lccl@{}}
\toprule
Port & Size & Direction & Description\tabularnewline
\midrule
\endhead
\texttt{slv\_addr\_base} & \texttt{HADDR\_SIZE} & Input  & Slave Base Address\tabularnewline
\texttt{slv\_addr\_mask} & \texttt{HADDR\_SIZE} & Input  & Slave Address Space Mask\tabularnewline
\bottomrule
\caption{Slave Interface Customisation Port}
\end{longtable}

\begin{longtable}[]{@{}lccl@{}}
\toprule
Port & Size & Direction & Description\tabularnewline
\midrule
\endhead
\texttt{slv\_HSEL}       & 1                    & Output & Bus Select\tabularnewline
\texttt{slv\_HADDR}      & \texttt{HADDR\_SIZE} & Output & Address\tabularnewline
\texttt{slv\_HWDATA}     & \texttt{HDATA\_SIZE} & Output & Write Data Bus\tabularnewline
\texttt{slv\_HRDATA}     & \texttt{HDATA\_SIZE} & Input  & Read Data Bus\tabularnewline
\texttt{slv\_HWRITE}     & 1                    & Output & Write Select\tabularnewline
\texttt{slv\_HSIZE}      & 3                    & Output & Transfer size\tabularnewline
\texttt{slv\_HBURST}     & 3                    & Output & Transfer Burst Size\tabularnewline
\texttt{slv\_HPROT}      & 4                    & Output & Transfer Protection Level\tabularnewline
\texttt{slv\_HTRANS}     & 2                    & Input  & Transfer Type\tabularnewline
\texttt{slv\_HMASTLOCK}  & 1                    & Output & Transfer Master Lock\tabularnewline
\texttt{slv\_HREADY}     & 1                    & Input  & Transfer Ready Input\tabularnewline
\texttt{slv\_HRESP}      & 1                    & Input  & Transfer Response\tabularnewline
\bottomrule
\caption{Slave Interface AHB-Lite Port}
\end{longtable}

\subsection{slv\_addr\_base}\label{slv_addr_base}

\texttt{slv\_addr\_base} is a \texttt{SLAVES} sized array of addresses, each
\texttt{HADDR\_SIZE} bits wide, defining the base address of each attached slave device.

\subsection{slv\_addr\_mask}\label{slv_addr_mask}
\texttt{slv\_addr\_mask} is a \texttt{SLAVES} sized array of \texttt{HADDR\_SIZE} bit wide
signals. Each \texttt{slv\_addr\_base} address is masked with the corresponding
\texttt{slv\_addr\_mask} value to define the addressable memory space of the attached slave.
Setting a bit of \texttt{slv\_addr\_mask} to '0' enables the corresponding address bit.

\ifdefined\MARKDOWN
  See section 'Address Space Configuration' for specific examples.
\else
  See section \ref{address-space-configuration} for specific examples.
\fi

\subsection{slv\_HSEL}\label{slv_hsel}

Slaves connected to the Slave Port must only respond to other signals on the bus when
\texttt{slv\_HSEL} is asserted (`1'). When \texttt{slv\_HSEL} is negated (`0') the
interface is idle and the connected Slaves must assert their \texttt{HREADYOUT} (`1').

\subsection{slv\_HADDR}\label{slv_haddr}

\texttt{slv\_HADDR} is the data address bus. Its size is determined by the
\texttt{HADDR\_SIZE} parameter. The connected master drives \texttt{slv\_HADDR}.

\subsection{slv\_HRDATA}\label{slv_hrdata}

\texttt{slv\_HRDATA} is the read data bus. Its size is determined by the
\texttt{HDATA\_SIZE} parameter. It is driven to the connected master.

\subsection{slv\_HWDATA}\label{slv_hwdata}

\texttt{slv\_HWDATA} is the write data bus. Its size is determined by the
\texttt{HDATA\_SIZE} parameter. The connected master drives \texttt{slv\_HADDR}.

\subsection{slv\_HWRITE}\label{slv_hwrite}

\texttt{slv\_HWRITE} is the read/write signal. \texttt{slv\_HWRITE} asserted (`1') indicates a
write transfer. The connected master drives \texttt{slv\_HWRITE}.

\subsection{slv\_HSIZE}\label{slv_hsize}

\texttt{slv\_HSIZE} indicates the size of the current transfer. The connected
master drives \texttt{slv\_HSIZE}.

\begin{longtable}[]{@{}ccl@{}}
\toprule
HSIZE & Size & Description\tabularnewline
\midrule
\endhead
\texttt{000} & 8bit    & Byte\tabularnewline
\texttt{001} & 16bit   & Half Word\tabularnewline
\texttt{010} & 32bit   & Word\tabularnewline
\texttt{011} & 64bits  & Double Word\tabularnewline
\texttt{100} & 128bit  & \tabularnewline
\texttt{101} & 256bit  & \tabularnewline
\texttt{110} & 512bit  & \tabularnewline
\texttt{111} & 1024bit & \tabularnewline
\bottomrule
\caption{Data Transfer Sizes}
\end{longtable}

\subsection{slv\_HBURST}\label{slv_hburst}

The burst type indicates if the transfer is a single transfer or part of
a burst. The connected master drives it.

\begin{longtable}[]{@{}ccl@{}}
\toprule
HBURST & Type & Description\tabularnewline
\midrule
\endhead
\texttt{000} & Single & Single access\tabularnewline
\texttt{001} & INCR   & Continuous incremental burst\tabularnewline
\texttt{010} & WRAP4  & 4-beat wrapping burst\tabularnewline
\texttt{011} & INCR4  & 4-beat incrementing burst\tabularnewline
\texttt{100} & WRAP8  & 8-beat wrapping burst\tabularnewline
\texttt{101} & INCR8  & 8-beat incrementing burst\tabularnewline
\texttt{110} & WRAP16 & 16-beat wrapping burst\tabularnewline
\texttt{111} & INCR16 & 16-beat incrementing burst\tabularnewline
\bottomrule
\caption{Burst Types (HBURST)}
\end{longtable}

\subsection{slv\_HPROT}\label{slv_hprot}

The data protection signals provide information about the bus transfer.
They are intended to implement some level of protection. The connected
master drives \texttt{slv\_HPROT}.

\begin{longtable}[]{@{}ccl@{}}
\toprule
Bit\# & Value & Description\tabularnewline
\midrule
\endhead
3 & 1 & Cacheable region addressed\tabularnewline
& 0 & Non-cacheable region addressed\tabularnewline
2 & 1 & Bufferable\tabularnewline
& 0 & Non-bufferable\tabularnewline
1 & 1 & Privileged access. CPU is not in User Mode\tabularnewline
& 0 & User access. CPU is in User Mode\tabularnewline
0 & 1 & Data transfer, always `1'\tabularnewline
\bottomrule
\caption{Data Protection Signals}
\end{longtable}

\subsection{slv\_HTRANS}\label{slv_htrans}

\texttt{slv\_HTRANS} indicates the type of the current data transfer.

\begin{longtable}[]{@{}ccl@{}}
\toprule
slv\_HTRANS & Type & Description\tabularnewline
\midrule
\endhead
\texttt{00} & IDLE & No transfer required\tabularnewline
\texttt{01} & BUSY & \emph{Not used}\tabularnewline
\texttt{10} & NONSEQ & First transfer of an data burst\tabularnewline
\texttt{11} & SEQ & Remaining transfers of an data burst\tabularnewline
\bottomrule
\caption{Data Transfer Type}
\end{longtable}

\subsection{slv\_HMASTLOCK}\label{slv_hmastlock}

The master lock signal indicates if the current transfer is part of a
locked sequence, commonly used for Read-Modify-Write cycles. The
connected master drives \texttt{slv\_MASTLOCK}.

\subsection{slv\_HREADYOUT}\label{slv_hreadyout}

The \texttt{slv\_HREADYOUT} signal reflects the state of the connected Master
Port's \texttt{HREADY} port. It is provided to support local slaves connected
directly to the Master's AHB-Lite bus. It is driven by the connected
master's \texttt{HREADY} port.

\textbf{Note:} \texttt{slv\_HREADYOUT} is not an AHB-Lite Signal.

\subsection{slv\_HREADY}\label{slv_hready}

\texttt{slv\_HREADY} indicates whether the addressed slave is ready to transfer
data or not. When \texttt{slv\_HREADY} is negated (`0') the slave is not ready,
forcing wait states. When \texttt{slv\_HREADY} is asserted (`0') the slave is
ready and the transfer completed. It is driven to the connected master's
\texttt{HREADYOUT} port.

\subsection{slv\_HRESP}\label{slv_hresp}

\texttt{slv\_HRESP} is the data transfer response, it can either be OKAY (`0') or
ERROR (`1'). It is driven to the connected master.
